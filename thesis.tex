\documentclass{article}

% Packages defined here
\usepackage{titlesec}
\usepackage{amsmath}
\usepackage{tikz-cd}
\usepackage{amsthm}

% pdf data defined here
\title{Title}
\author{Author} \date{ }
% Abbreviations defined here
\newcommand{\spec}[1]{\operatorname{Spec}(#1)}
\newcommand{\Hom}{\operatorname{Hom}}
\newcommand{\sheafhom}{\mathcal{H}\mathit{om}}
\newcommand{\cat}{\mathcal C} % This is going to be the Tannakian category we are looking for

\newtheorem{theorem}{Theorem}
\newtheorem{lemma}{Lemma}


\begin{document}

\maketitle

\tableofcontents

\newpage

\section{Introduction}
For a fixed scheme $S$, let $X$ be a scheme over $S$ and $G$ be a group scheme over $S$ with the product map 
$\mu:G \times _S G \rightarrow G$, acting on $X$ via $\sigma: G \times _S X \rightarrow X$. We shall call the pair
$(\mathcal F, \phi_\mathcal F)$ an equivariant sheaf, when $\mathcal F$ on $X$ is a quasi-coherent sheaf of 
$\mathcal O_X$-modules, and $\phi_{\mathcal F}: \sigma ^* \mathcal F \rightarrow p_2^* \mathcal F$ is an isomorphism 
satisfying the cocycle condition $$p_{23}^* \phi \circ (1_G \times _S \sigma)^*\phi = (m \times _S 1_X)^*\phi.$$ 
This cocycle condition is equivalent to saying that the following diagram commutes:

\begin{center}
\begin{tikzcd}
  & \left [ \sigma \circ (1_G \times \sigma) \right ]^* \mathcal F \arrow[ld,equal]\arrow[rd,"(1_G \times \sigma)^*\phi_\mathcal F"]& \\
  \left [ \sigma \circ (\mu \times 1_X) \right ]^* \mathcal F  \arrow[d,"(\mu \times 1_X)^*\phi_\mathcal F"]&   & \left [ p_2 \circ (1_G \times \sigma) \right ]^* \mathcal F\arrow[d,equal]\\
  \left [ p_2 \circ (\mu \times 1_X) \right ]^* \mathcal F \arrow[rd,equal] &   & \left [ \sigma \circ (p_{23}) \right ]^* \mathcal F\arrow[ld,"p_{23}^*\phi_\mathcal F"] \\
    & \left [ p_2 \circ p_{23} \right ]^* \mathcal F  & 
\end{tikzcd}
\end{center}


A morphism $f:(\mathcal F,\phi _\mathcal F) \rightarrow (\mathcal G, \phi_\mathcal G)$ between two equivariant sheafs
is defined to be a morphism $f:\mathcal F \rightarrow \mathcal G$ of sheaves of $\mathcal O_X$-modules,
such that the following diagram commutes
\begin{center}
  \begin{tikzcd}
    \sigma ^* \mathcal F \arrow[r,"\phi_\mathcal F"] \arrow [d,"\sigma^* f"]   & p_2^* \mathcal F \arrow[d,"p_2^*f"]\\
    \sigma ^* \mathcal G \arrow[r,"\phi_\mathcal G"]              & p_2^* \mathcal G
  \end{tikzcd}
\end{center}

We see that a composition of a morphism of equivariant sheaf is well-defined, for if we have equivariant sheafs
$(\mathcal F,\phi _\mathcal F)$, $(\mathcal G,\phi _\mathcal G)$ and $(\mathcal H,\phi _\mathcal H)$ and morphisms
$f:(\mathcal F,\phi _\mathcal F) \rightarrow (\mathcal G, \phi_\mathcal G)$ and
$g:(\mathcal G,\phi _\mathcal G) \rightarrow (\mathcal H, \phi_\mathcal H)$ then the composition of the morphisms 
$g \circ f : (\mathcal F,\phi _\mathcal F) \rightarrow (\mathcal H,\phi _\mathcal H)$ is also a morphisms of
equivariant sheafs.

We see that $(\mathcal F,\phi_{\mathcal F})$ is a quasi-coherent sheaf if and only if $\mathcal F$ is an equivariant
object in the fibered category of quasi-coherent sheafs. This means that under the setup 
$p:\mathcal Qcoh/k \rightarrow Sch/k$ and $\mathcal F \in \mathcal Qcoh(X)$ we have that $\mathcal F$ is a 
$G$-equivariant sheaf on $X$ if and only if for every quasi-coherent sheaf $\mathcal G$ over $T \rightarrow \spec k$,
we have a naturalaction of $\Hom(T,G)$ on $\Hom(\mathcal G,\mathcal F)$ such that
\begin{enumerate}
  \item For every arrow $\phi:\mathcal G_1 \rightarrow \mathcal G_2$ in $\mathcal Qcoh/k$ mapping to a morphism of 
    schemes $f:T_1 \rightarrow T_2$, the induced function 
    $\phi^*: \Hom(T_2,\mathcal F) \rightarrow \mathcal (T_1,\mathcal F)$ is equivariant with respect to the 
    homomorphism $\Hom(f,G):\Hom (T_2,G) \rightarrow \Hom (T_1,G)$.
  \item The function $\Hom (\mathcal G,\mathcal F) \rightarrow \mathcal (T,X)$ induced by the functor $p$ is 
    $\Hom(T,G)$ equivariant.
\end{enumerate}
A $G$-equivariant morphism $f:\mathcal F \rightarrow \mathcal G$ of quasi-coherent sheafs
$\mathcal F$ and $\mathcal G$ is a morphism of sheafs, such that for each quasi-coherent quasi-coherent sheaf
$\mathcal V$ over a $k$-scheme $Y$, the natural transformation of functors
$\tilde f(\mathcal V): \Hom(\mathcal V,\mathcal F) \Rightarrow \Hom(\mathcal V,\mathcal G)$ induced by $f$ is a 
a $\Hom(Y,G)$-equivariant map.

\section{Equivariant Vector Bundles} 

Let $k$ be a perfect field with algebraic closure $\bar k$ and let $X$ be a complete connected reduced scheme over $k$.
Denote by $G$, the constant profinite group $Gal(\bar k / k)$ over $\bar k$. Observe that $G$ acts naturally on 
$\bar X := X \times _k \spec {\bar k}$. Denote the natural action of $G$ on $X$ by 
$\sigma: G \times _{\bar k} \bar X \rightarrow \bar X$. We study essentially finite vector bundles over $\bar X$ with 
$G$-equivariant structures.

\subsection{Equivariant Vector Bundles}
\label{sec:Equivariant Vector Bundles}

Let $\cat$ be the full subcategory of the category of equivariant sheafs, whose objects are of the form $(V,\phi_V)$,
where $V$ is an essentially finite vector bundle. We shall show that $\cat$ forms a Tannakian category. 

\begin{lemma}
  Both $p_2:G \times X \rightarrow X$ and $\sigma:G \times X \rightarrow X$ are flat.
\end{lemma}
\begin{proof}
  Since $G$ is a limit of finite groups which are flat, we have that the structure map $G \rightarrow \spec {\bar k}$ 
  is flat. So we have that the base change of this map by $\bar X \rightarrow \spec {\bar k}$ is flat, so 
  $p_2: G \times \bar X \rightarrow \bar X$ is flat.

  Now we want to show that the group action $\sigma$ is flat. For any finite Galois extension $K$ of $k$, we set
  $G_K$ to be the constant group $Gal(K/k)$ over $\spec K$ and $X_K = X \times _{\spec k} \spec K$. We see that
  $G_K$ is finite and flat over $\spec K$. For a tower of Galois extensions $K' \supset K \supset k$, we have that
  $$G_{K'} \rightarrow G_K \times_K \spec {K'} \xrightarrow {p_1} G_K.$$ 
  Similarly we have a map $X_{K'} \rightarrow X_K$. We see that the limits of $G_K$ and $X_K$ over all finite Galois
  extensions $K$ of $k$ are $G$ and $\bar X$ respectively. We similarly see that the limit of $G_K \times_K X_K$ 
  over all finite Galois $K\supset k$ is $G \times_{\bar k} \bar X$. Now we have the natural action 
  $\sigma_K: G_K \times_K X_K \rightarrow X_K$ of $G_K$ on $X_K$. This action is flat and we have that this is a map
  of inverse systems whose limit is the map $\sigma: G \times \bar X \rightarrow \bar X$, so $\sigma$ is flat.
\end{proof}
\begin{lemma}
  For equivariant vector bundles $(V_i,\phi_{V_i})$ and $(W_i,\phi_{W_i})$ for $i=1,2$ and maps
  $f:(V_1,\phi_{V_1}) \rightarrow (V_2,\phi_{V_2})$ and $g:(W_1,\phi_{W_1}) \rightarrow (W_2,\phi_{W_2})$, 
  $\Hom (V_1,\phi_{V_1}){(W_1,\phi_{W_1})}$ forms an abelian group and $f$ and $g$ induce a homomorphism of groups.
\end{lemma}
\begin{proof}
  Let $h_1$ and $h_2$ be two morphism from $\Hom ((V_1,\phi_{V_1}),(W_1,\phi_{W_1}))$. Then $h_1 + h_2$ also
  defines an equivariant morphism from $(V_1,\phi_{V_1})$ to $(W_1,\phi_{W_1})$ as 
  \begin{align*}
    (p_2^*(h_1 + h_2)) \circ \phi_{V_1} &= p_2^*(h_1) \circ \phi_{V_1} + p_2^*(h_2) \circ \phi_{V_1} \\
                                        &= \phi_{V_2} \circ \sigma^* (h_1) + \phi_{V_2} \circ \sigma^* (h_2) \\
                                        &= \phi_{V_2} \circ \sigma^* (h_1 + h_2).
  \end{align*}
  So we have that $\Hom ((V_1,\phi_{V_1}),(V_2,\phi_{V_2}))$ is an abelian group.
  Now we want to show that $f$ induces a homomorphism 
  $$\tilde f:\Hom ((V_2,\phi_{V_2}),(W_1,\phi_{W_1}))\rightarrow \Hom ((V_1,\phi_{V_1}),(W_1,\phi_{W_1})).$$
  For $h \in \Hom ((V_2,\phi_{V_2}),(W_1,\phi_{W_1}))$, we have that $\tilde f (h) = h \circ f$ so we have for 
  $h_1,h_2 \in \Hom ((V_2,\phi_{V_2}),(W_1,\phi_{W_1}))$, 
\begin{align*}
  \tilde f (h_1 + h_2) &= (h_1 + h_2) \circ f \\
                       &= h_1 \circ f + h_2 \circ f \\
                       &= \tilde f (h_1) + \tilde f (h_2)
\end{align*}
so $\tilde f$ is a homomorphism. 

Similarly the map $\tilde g$ induced by $g$ on the set of morphisms is also a homomorphism of abelian groups.

\end{proof}

\begin{lemma}
  For two equivariant vector bundles $(V_1,\phi_{V_1})$ and $(V_2,\phi_{V_2})$ any equivariant morphism
  $f:(V_1,\phi_{V_1})\rightarrow (V_2,\phi_{V_2})$ admits a kernel and a cokernel.
\end{lemma}
\begin{proof}
  From the hypothesis, we have a map $f:V_1 \rightarrow V_2$ of essentially finite vector bundles. Since the 
  category of essentially finite vector bundles is Abelian, we have that $f$ has a kernel $j_K:K\rightarrow V_1$ 
  and a cokernel $q_C:V_2 \rightarrow C$. Since $\sigma$ and $p_2$ are exact, we have the following diagram which 
  commutes
  \begin{center}
    \begin{tikzcd}
      0 \arrow[r] \arrow[d] & \sigma^* K \arrow[r,"\sigma^* j_K"] \arrow[d,dotted,"\phi_K"] & \sigma^* V_1 \arrow[r,"\sigma^* f"] \arrow[d,"\phi_{V_1}"] & \sigma^* V_2 \arrow[r,"\sigma^* q_C"] \arrow[d,"\phi_{V_2}"] & \sigma^* C \arrow[r] \arrow[d,dotted,"\phi_C"] & 0 \arrow[d]\\
      0 \arrow [r] & p_2^* K \arrow[r,"p_2^* j_K"] & p_2^* V_1 \arrow[r,"p_2^* f"] & p_2^* V_2 \arrow[r,"p_2^* q_C"] & p_2^* C \arrow[r] & 0
    \end{tikzcd}
  \end{center}
  where $\phi_k$ is induced by the universal property of $p_2^* K$ being the kernel of $p_2^* f$, and $\phi_C$ is 
  induced by the universal property of $\sigma_K ^* C$ being the cokernel of $\sigma ^* f$. 
  By extending the rows on the left and right of the sequence by zeros, and using five lemma on 
  the five columns to the left and the five columns to the right,
  we see that both $\phi_K$ and $\phi_C$ are isomorphisms.

  Now we verify that both $\phi_K$ and $\phi_C$ indeed satisfy the hexagon. We have 
  \begin{align*}
    (1_G\times \sigma)^*j_K\circ p_{23}^* \phi_{K} \circ (1_G \times _S \sigma)^*\phi_{K} =&\\
    p_{23}^* \phi_{V_1} \circ (1_G \times _S \sigma)^*\phi_{V_1} \circ (1_G\times \sigma)^*j_K&= \\
    (m \times _S 1_X)^*\phi_V{_1}\circ (1_G\times \sigma)^*j_K&= \\
    (1_G\times \sigma)^*j_K\circ (m \times _S 1_X)^*\phi_V{_1}.
  \end{align*}
  Since $(1_G\times \sigma)^*j_K$ is monic, we have that the cocycle condition for $\phi_K$ is satisfied. Thus we have
  That $(K,\phi_K)$ is an equivariant vector bundle. Similarly $(C,\phi_C)$ is an equivariant vector bundle.
\end{proof}

So we see that $\cat$ is an Abelian category. Now we shall show that we have a tensor product in $\cat$, which makes
it a Tannakian category

Define $\otimes: \cat \times \cat \rightarrow \cat$ by 
$(V,\phi_V) \otimes (W,\phi_W) = (V\otimes W,\phi_V \otimes \phi_W)$. To check that the functor is well defined,
we have to check that $\phi_V \otimes \phi_W$ satisfies the cocycle condition:
\begin{align*}
  p_{23}^* (\phi_V \otimes \phi_W) \circ (1_G \times_S \sigma) ^* (\phi_V \otimes \phi_W) \\
  (p_{23}^* \phi_V \circ p_{23}^* \phi_W) \otimes ((1_G \times_S \sigma) ^* \phi_V \circ (1_G \times_S \sigma) ^* \phi_W) \\
  (p_{23}^* \phi_V \circ (1_G \times_S \sigma) ^* \phi_V) \otimes (p_{23}^* \phi_W \circ (1_G \times_S \sigma) ^* \phi_W) \\
  = (m \times 1_X)^* \phi_V \otimes (m \times 1_X)^* \phi_W\\
  = (m \times 1_X)^* \phi_V \otimes \phi_W.
\end{align*}
Thus $(V,\phi_V) \otimes (W,\phi_W) = (V\otimes W,\phi_V \otimes \phi_W)$ is a equivariant vector bundle.

Now we shall show that the sheaf hom $\Hom(V_1,V_2)$ can naturally be given a $G$-equivariant structure.

\begin{lemma}
  If $(V_1,\phi_{V_1})$ and $(V_2,\phi_{V_2})$ are $G$-equivariant sheafs, then the sheaf $\sheafhom (V_1,V_2)$ has a 
  natural $G$-equivariant structure.
\end{lemma}

\begin{proof}
  On $\mathcal Qcoh/k$, we take the functor $F:\mathcal Qcoh \rightarrow Sets$ defined, for a sheaf
  $\mathcal V$ over a scheme $Y$, by setting $F(\mathcal V)$ to be the set of all pairs $(f,\tilde f)$ where $f$ 
  is a morphism of $k$-schemes $f:Y\rightarrow X$, and $\tilde f$ is a homomorphism of $\Gamma(Y,\mathcal O_Y)$-modules
  $\Gamma(Y,F)$ to $\Hom(\Gamma(Y,V_1),\Gamma(Y,V_2))$ and for a morphism of sheaves 
  $\tilde g:\mathcal F \rightarrow \mathcal G$ over the map $g:Y_1 \rightarrow Y_2$ of $k$-schemes, defining the map
  $F(\tilde g):F(\mathcal G) \rightarrow F(\mathcal F)$ by sending $(f,\tilde f)$ to 
  $(f\circ g,\tilde f \circ \tilde g)$. We see that $F$ defines a presheaf of sets on 
  $\mathcal Qcoh/k$, whose sheafification under the big Zariski topology is isomorphic to the sheaf
  $\Hom(-,\sheafhom (V_1,V_2))$. Now we have an action of $G$ on the preasheaf $F$, given by taking a 
  $g\in \Hom(Y,G)$ and $(f,\tilde f) \in F(Y)$, and defining 
  $$g\cdot (f,\tilde f) = ((g\cdot f), (g\cdot \tilde f\cdot g^{-1})).$$
  Thus $G$ has an action on the sheaf $\Hom(-,\sheafhom(V_1,V_2))$, giving us a $G$-equivariant structure on 
  $\sheafhom(V_1,V_2)$.
\end{proof}

\begin{lemma}
  Let $\mathcal E,\mathcal F$ and $\mathcal G$ be $G$-equivariant sheafs on $X$. Then we have
  $$\Hom_\cat(\mathcal E \otimes \mathcal F, \mathcal G) \cong\Hom_\cat(\mathcal E,\sheafhom(\mathcal F,\mathcal G)).$$
\end{lemma}
\begin{proof}
  From Yoneda lemma, we have that the set $\Hom_\cat(\mathcal E \otimes \mathcal F,\mathcal G)$ is the same as 
  the set of all $G$-equivariant natural transformations 
  $F:\Hom(-,\mathcal E \otimes \mathcal F) \Rightarrow \Hom(-,\mathcal G)$ of
  the set valued functors $\Hom(-,\mathcal E \otimes \mathcal F)$ to $\Hom(-,\mathcal G)$ on $\mathcal Qcoh/k$.
  Similarly, $\Hom_\cat(\mathcal E ,\sheafhom(\mathcal F,\mathcal G))$ is the same as 
  the set of all $G$-equivariant natural transformations 
  $F:\Hom(-,\mathcal E) \Rightarrow \Hom(-,\sheafhom(\mathcal F,\mathcal G))$ of
  the set valued functors $\Hom(-,\mathcal E)$ to $\Hom(-,\sheafhom(\mathcal F,\mathcal G))$ on $\mathcal Qcoh/k$.

  Now we define preasheaf of sets $F_1:\mathcal Qcoh/k \rightarrow Sets$ by assigning to each quasi-coherent sheaf
  $\mathcal V$ over the $k$-scheme $Y$, the set of $(f,\tilde f)$ such that $f:Y \rightarrow X$ is a morphism of 
  $k$-schemes and $\tilde f$ is a $\Gamma(Y,\mathcal O_Y)$-module morphism of $\Gamma(Y,\mathcal V)$ to 
  $\Gamma(Y,f^* (\mathcal E \otimes \mathcal F))$. For a morphism $\tilde g:\mathcal V_1 \rightarrow \mathcal V_2$ 
  over the morphism of schemes $g:Y_1 \rightarrow Y_2$, we set $F_1(g)$ to be the map which takes the pairs
  $(f,\tilde f)$ and maps it to $(f \circ g,\tilde f \circ \tilde g )$.

  We define preasheaf of sets $F_2:\mathcal Qcoh/k \rightarrow Sets$ by assigning to each quasi-coherent sheaf
  $\mathcal V$ over the $k$-scheme $Y$, the set of $(f,\tilde f)$ such that $f:Y \rightarrow X$ is a morphism of 
  $k$-schemes and $\tilde f$ is a $\Gamma(Y,\mathcal O_Y)$-module morphism of $\Gamma(Y,\mathcal V)$ to 
  $\Hom(\Gamma(Y,f^*(\mathcal F)),\Gamma(Y,f^*\mathcal G))$.

  Now we have that $G$-equivariant natural transformations of $F_1$ to $\Hom(-,\mathcal G)$ are isomorphic to the
  $G$-equivariant natural transformations of $\Hom(-,\mathcal E)$ to $F_2$. Upon sheafification of both $F_1$ and
  $F_2$ under the big Zariski topology, we get the required isomorphism.
\end{proof}

Let $\bar x: \spec {\bar k} \rightarrow X$ be a geometric point. We see that the fiber functor of $\cat$ associated to 
$\bar x$ is an exact $\bar k$-linear faithful tensor functor and $\operatorname{End}(\mathcal O_{\bar X}) = \bar k$. 
Thus $\cat$ along with $\bar x$ forms a Tannakian category with dual $\pi(X,\bar x)$. 

\subsection{Results on $\pi(X,\bar x)$}

Now we compare Nori's fundamental group with the aforementioned fundamental group.
\begin{lemma}
  Let $\bar f:\bar X\rightarrow \bar Y$ be a morphism of $\bar k$-varieties, then there exists a 
  finite field extension $K$ of $k$, varieties $X$ and $Y$ over $K$, and a morphism $f:X \rightarrow Y$ of 
  $K$-varieties, such that base change of $f$ to $\bar k$ is isomorphic to $\bar f$.
\end{lemma}
\begin{proof}
  Let $U_i = \spec {A_i}$ be an affine open cover of $\bar Y$ and let $V_{ij} = \spec {B_{ij}}$ be an affine open cover
  of $\bar f ^{-1} (U_i)$. Now each $A_i$, $A_i \times _{\bar Y} A_j$, $B_{ij}$ and $B_{ij} \times _{\bar X} B_{kl}$ 
  is a finite type $\bar k$-algebra and the maps between them are polynomial maps so we can take the field extension 
  $K$ of $k$, which we get by adjoining coefficients of polynomials defining the affine open sets and the coefficients
  of the polynomial map $\bar f$, restriced to $\bar f : \spec B_{ij} \rightarrow {A_i}$. Since both $\bar X$ and 
  $\bar Y$ are quasi-compact, we may take finitely many $U_i$ and $V_{ij}$, so $K$ is a finite extension of $k$. This
  gives us patching data to construct a map $f:X \rightarrow Y$ of varieties over $K$, and by construction, we have 
  the the pullback of this map to $\bar k$ is $\bar f$. 
\end{proof}
\begin{theorem}
  Let $X$ be a complete connected reduced scheme over $k$. Nori's fundamental group $\pi^N(\bar X, \bar x)$ is a
  closed subgroup of $\pi(X,\bar x)$.
\end{theorem}

\begin{proof}
  To show the result, we recall that a tensor functor $F:T_1 \rightarrow T_2$ between two Tannakian categories $T_1$ 
  and $T_2$ induces a group-scheme homomorphism $\pi(F):\pi(T_2)\rightarrow \pi(T_1)$. This homomorphism is a closed 
  immersion if and only if every object of $T_2$ is isomorphic to a subquotient of an object of the form of 
  $F(t)$, where $t$ is an object of $T_1$.

  Now we have a tensor functor $\omega:\cat \rightarrow EF$, given by $\omega((V,\phi_V)\xrightarrow f(W,\phi_W)) = 
  V\xrightarrow f W$, where $EF$ is the category of essentially finite vector bundles on $\bar X$. We shall show that 
  every essentially finite vector bundle $V$ is a subquotient of a vector bundle $W$, where $W$ admits a 
  $G$-equivariant structure. So we fix an essentially finite vector bundle $\bar V$ on $\bar X$. We have the scheme 
  $\bar Y =\spec {Symm (\bar V)}$ and a map $\pi :\bar  Y \rightarrow \bar X$, where $Symm(\bar V)$ is the symmetric 
  algebra of $\bar V$. 

  Now from the previous lemma, we have a finite extension $K$ of $k$, a variety $V$ over $K$, and a map 
  $\tilde \pi : V \rightarrow X \times _k \spec K$ such that base change of  $\tilde \pi$ to $\bar k$ is isomorphic
  to $\pi$. So $\bar V$ is a pullback of the vector bundle $V$ over $X \times_k \spec K$.

  This give us a vector bundle $p _{1*}V$ on $X$, where $p_1$ is the projection of $X \times _k \spec K$ to $X$. 
  From the counit functor we get the map $p_1^*p _{1*}V$ surjects onto $V$ (since $p_1$ is an affine map). 
  Thus the pullback of the vector bundle $p_{1*}V$ on $X$ along $\spec {\bar k} \rightarrow \spec k$ give us a vector
  bundle $W$ on $\bar X$, such that $V$ is a quotient of $W$. Since $W$ is the pullback of a vector bundle along a
  $G$-torsor, we have that $W$ is a $G$-equivariant bundle. So we have that the map 
  $\pi(\omega):\pi^N(\bar X, \bar x) \rightarrow \pi(X,\bar x)$ is a closed immersion.
\end{proof}

Let $\mathcal Rep$ be the category of finite dimensional $\bar k$-representaions of the absolute Galois group of
$k$. Then we have a functor $F:\mathcal Rep \rightarrow \cat$, which is defined by taking a morphism 
$f:V \rightarrow W$ in $\mathcal Rep$ and setting $F(f)=Id_{\bar X} \times f$. We see that this is a $k$-linear tensor
functor.

\begin{theorem}
  The homomorphism $\pi(F):\pi(X,\bar x) \rightarrow G$ defined by the functor $F$ is a faithfully flat morphism of 
  schemes. 
\end{theorem}

\begin{proof}
  We know that $\pi(F)$ is faithfully flat if and only if $F$ is fully faithful and every object $V$ of 
  $\mathcal Rep$, the subobjects of $F(V)$ are isomorphic to the image of a subobject of $V$.

  So we fix a finite dimensional $\bar k$-representaion $V$ of $G$, and we get an equivariant $G$-bundle 
  $\bar X \times V$ on $\bar X$. This corresponds to the sheaf $\mathcal F$, which is isomorphic to 
  $\oplus \mathcal O_{\bar X}$ with a $G$-equivariant structure on $\bar X$. Now let $\mathcal V$ be a sub object of 
  $\mathcal F$ in the category $\cat$. So we want to show that $\mathcal V$ is also of the form 
  $\oplus \mathcal O_{\bar X}$. Since $\mathcal V$ is an essentially finite vector bundle, there exists a 
  finite group-scheme $H$ over $\bar k$ and an $H$-torsor $f:Y \rightarrow \bar X$, such that 
  $f^*(\mathcal V) \cong \oplus \mathcal O_Y$. Now $f^*(\mathcal F) \cong \oplus \mathcal O_Y$ with the trivial
  $H$-equivariant structure, so $f^*(\mathcal V)$ too has a trivial $H$-equivariant structure, so 
  $\mathcal V \cong f_*f^*(\mathcal V) ^H \cong \oplus \mathcal O_{\bar X}$
\end{proof}

\begin{theorem}
  From the two homomorphisms above, we get a short exact sequence
  $$0 \rightarrow \pi ^ N(\bar X,\bar x) \rightarrow \pi(X,\bar x) \rightarrow G \rightarrow 0.$$
\end{theorem}

\begin{proof}
  We see that $\omega \circ F$ take a representaion $V$ of $G$ and gives a trivial vector bundle on $\bar X$ of the
  same rank, so we have the above sequence is a complex. Now objects of $\cat$ which give a trivial vector bundle 
  under $\omega$ are precisely of the form $\oplus \mathcal O_{\bar X}$ with a $G$-equivariant structure which is 
  isomorphic to an object in the image of $F$.
\end{proof}

\end{document}
